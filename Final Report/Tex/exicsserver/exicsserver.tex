\chapter{ExICS Server}

\label{ch:exicsserver}

Having designed a the API to be used to communicate between server and client, and implemented the API server capable of providing the exam status and seating plan information, it was neccesary to begin implementing the main ExICS server.

\section{ExICS Server Architecture}

The design of the ExICS server is relatively simple with four main components; The main client-facing server logic to handle events, a ``class'' holding system state, a ``class'' offering and handling both logging and persistent storage capabilities, and authentication logic.

\subsection{ExICS Logic}

The ExICS server uses the ``ws'' Node.JS package\cite{einarosWS} to provide the fundemental low-level websockets used to communicate between server and clients.

When the ExICS server ``app'' is run, a websockets server is started and a callback function to be executed by Node when a client connects is assigned.

Node.JS, built on Chrome's JavaScript runtime means that it makes full use of the event driven callback design of javascript, requiring and enabling full use of closures in the code.

The callback function registered for callback upon connection of a client to the websocket server, when called, creates a closure capturing the state of that connection, allowing all variables used, such as the ``authenticated'' boolean and string holding the username that was authenticated can be accessed seamlessly within the closure and modified safely, as they are unique to that specific client's callback closure and inaccessible from other client code.

The ``onConnection'' callback function for each client connection contains a number of members, the ``auth'' boolean as well as a string containing the authenticated user's username.  It also overrides the websocket server methods, ``onMessage'' and ``onClose''.

When a user first connects, the closure defining ``their'' callbacks is created by Node.JS creating the context for all future interactions with that user.  Initially ``auth'' is false and there is no string for the users authorised username, and the event is logged in the ExICS system log, explained in section \ref{subs:exics_logging}.

When the user sends a message to the server over the established socket, the ``onMessage'' callback is invoked to process the message.  If the user is not authenticated, i.e ``auth'' is false, the message's header and the type parameter is checked.  The only valid message to be received when not authorised is a PROTOCOL\_HANDSHAKE message, as defined in section \ref{sec:api_design}.  If any other message type, or an unknown message type is received, the server forcibly closes the websocket connection, disconnecting the user.

If the received message is a PROTOCOL\_HANDSHAKE message, the server executes the ``ldapAuthenticate'' function with the provided credentials, as explained in section \ref{subs:exics_auth}.  Providing authentication of credentials with the college LDAP server is successful, the server updates the connection's closure's ``auth'' and username variables to reflect the successful authentication.  Should authentication fail, the server closes the connection to the user.

Given the the user is now authenticated, they are added to the main system state in the ``exICSData'' class, as explained in section \ref{subs:exics_data}, and shared with other connected clients. Any subsequent messages received are then handled in the ``onMessage'' callback of the code, where again the message header type is switched upon \cite{jsswitch} to select the correct method of the ``exICSData'' object to call.

When the user sends a TERMINATE\_CONNECTION message, or disconnects manually, the ``onClose'' callback is executed and the user is removed from the system list of clients and the closure containing the client's socket is cleaned up, ending their interaction with the server.

\subsection{ExICS System State Class - ``ExICSData''}
\label{subs:exics_data}

The ``ExICSData'' class of the ExICS server is used to store and manipulate the system state.  Because of the way every new connection invokes a new instance of the ``onConnected'' callback and creates its own closure, data defined in and belonging to that closer, is out of scope for all other instances of the callback, and so all other connections.

As a result, it is necessary to store all data which is shared between all clients, the state of the overall system in a globally accessible way, such that all client contexts can read from and modify it.

For this reason, the system state of ExICS is stored in a standalone ``singleton'' instance which can be fetched and mutated from anywhere, by any connection context.  Each ``onConnection'' instance belonging to a client's socket on the server obtains a reference to the ExICSData object, and so when a change is made, it is visible to all other connection contexts.  The Javascript singleton pattern differs substantially from other language implementations such as Java or C++, and is explained in more detail in section \ref{subsubs:jsSingleton}.

The ExICSData singleton object contains all the members necessary to represent and allow manipulation of the system state.

There is a ``clientsConnected'' counter which holds the integer value of the number of clients currently connected to the server, even if unauthenticated.

In order to send data to all connected clients, there is an object representing connected clients.  The object features key value pairs, where the client's username is the key and the value is another object holding the socket instance through which data can be pushed, and the room in which the user is based.

Javascript objects behave much as a Hashmap or Map object in other languages such as Java behave, where the key is a hashable object itself, such as an int or string, mapping to a value object which could be an array, string, int etc., or another object itself.

Examinations scheduled for the current exam session period, the morning or afternoon of the day are stored in a Javascript object.  The room numbers exams are taking place in act as the keys with an array of exam objects, as explained in section \ref{sec:exics_objs}, as the value.

The objects and variables holding the state of the system are private members of the singleton object, along with some private methods used in maintaining the system state and for debugging purposes such as generating static date objects.

The singleton then offers a large number of publicly callable methods which can be called to perform actions upon the system state such as to start and exam, find the number of connected clients in a particular room, or invoke a system data refresh.

\subsubsection{Javascript Singleton Pattern}
\label{subsubs:jsSingleton}

\FloatBarrier

Whilst Javascript is an incredibly flexible and dynamic object oriented language, it is class-less meaning that ``classes'' in a traditional sense cannot be defined and instantiated.

Everything is in object and so can behave as if it was the instance of a class, and yet nothing is instantiated from class definitions.  Whilst it is not possible to define a ``class'' per-se, it is possible to create objects which \textit{behave} as you would expect a traditional class, with callable methods\cite{jsClasses}.

Due to the flexibility of the language there are many ways to create such objects.  Consider the code in figure \ref{fig:jsObjectInst};  this code creates an object called apple with properties ``type'', and ``colour'', as well as a callable function which returns the type and colour of the apple.  With this code, you don't need to and in fact \textit{can't} create an instance of this apple ``class'', it already exists and can be used, as shown in figure \ref{fig:usingAppleClass}.

\begin{figure}
\lstset{language=JavaScript}
\begin{lstlisting}[tabsize=2,breaklines=true]
	var apple = {
			type: "macintosh",
			colour: "red",
			getInfo: function () {
					return this.colour + ' ' + this.type + ' apple';
			}
	}
	\end{lstlisting}
	\caption{Instantiation of an Object in Javascript}
	\label{fig:jsObjectInst}
\end{figure}

\begin{figure}
\lstset{language=JavaScript}
\begin{lstlisting}[tabsize=2,breaklines=true]
	//Changes apple colour to ``reddish''
	apple.colour = "reddish";
	//Prints ``reddish macintosh apple''
	console.log(apple.getInfo());
	\end{lstlisting}
	\caption{Use of the Created ``Class''}
	\label{fig:usingAppleClass}
\end{figure}

Objects such as this apple are also known as a singleton; only one instance of this ``class'' can ever exist.  You can't create more of them without redefining a whole new instance as with figure \ref{fig:jsObjectInst}.  With Javascript having no classes, the notion of singleton's no longer makes sense, as everything is inherently a singleton to begin with.

With the method shown in figure \ref{fig:jsObjectInst}, everything defined is accessible though; for example the ``colour'' property of the apple can be changed by directly accessing the property and assigning it.  This is not good object-oriented design as the property of abstraction and encapsulation are violated.  As a result, a different design pattern, as shown in figure \ref{fig:nodeJSsingleton}, can be used to instantiate an object, accessible from anywhere which has both public and private members.

\begin{figure}
	\lstset{language=JavaScript}
	\begin{lstlisting}[tabsize=2,
			breaklines=true]
	//Node.JS singleton
	var mySingleton = (function () {
		// Instance stores a reference to the Singleton
		var instance;

		function init() {
			// Singleton
			// Private methods and variables
			var foo = true;

		bar: function bar(arg){
			...
		}
		...
			return {
					// Public methods and variables
					publicBar: function publicBar(){
				...
			},	
			};
		};
		return {
			getInstance: function () {
				if ( !instance ) {
					instance = init();
				}
				return instance;
			}
		};
	})();

	exports.mySingleton = mySingleton;
	\end{lstlisting}
	\caption{Singleton Pattern in Javascript (Node.JS)}
	\label{fig:nodeJSsingleton}
\end{figure}

This design pattern makes use of Javascript's closure and variable scoping properties to generate what would be recognised as a traditional ``class'' as found in other programming languages in the way it behaves.

\FloatBarrier

Importing ``mySingleton'' into another Node.JS using the statement

\FloatBarrier

\lstset{language=JavaScript}
\begin{lstlisting}[tabsize=2,
		breaklines=true]
	var ExICSData = require('./ExICSSystemData').ExICSData;
\end{lstlisting}

\FloatBarrier

exposes the ``getInstance()'' method of the ``class'' which can be called with

\FloatBarrier

\lstset{language=JavaScript}
\begin{lstlisting}[tabsize=2,
		breaklines=true]
	var systemData = ExICSData.getInstance();
\end{lstlisting}

\FloatBarrier

to return the actual instance, the object which holds the methods and private members we want associated with the ExICSData ``singleton''.

This returned instance has all of the publicly callable methods defined, allowing the calling code to call such methods to modify system state such as

\FloatBarrier

\lstset{language=JavaScript}
\begin{lstlisting}[tabsize=2,
		breaklines=true]
switch (parseInt(parsedMessage["header"]["type"])) {
	case PACKET_TYPE.SYSTEM_STATE:
		serverUtils.log("Received request for Current System State from " + username);
		systemData.pushSystemState(socket);
		break;

	case PACKET_TYPE.CHANGE_ROOM:
		serverUtils.log("Received Change Room Request from User " + username + " to room " + parsedMessage["payload"]["room"]);
		systemData.changeRoom(username, parsedMessage["payload"]["room"]);
		break;

	...
\end{lstlisting}

\FloatBarrier

Private members such as the data structures holding the state information are not accessible due to the Javascript scoping restrictions, but \textit{are} accessible from within the code defining the public methods themselves.

Due to the single-threaded execution model used by Node.JS, it is not neccessary to define threadsafe state mutators such as start or stop exam, as the execution model makes all data structures inherently safe, as explained in section \ref{subs:exics_backend_server}.

\subsubsection{Data Synchronisation}

The ExICSData object is designed to provide an easy and safe way to share system state between all connected clients, but when the server starts, there is no data populated.

Whilst system state, such as connected clients and client rooms are populated and updated at runtime as users connect and disconnect, the exams running during a period of time need to be populated from the exam data API explained in section \ref{ch:apiwrappers}.

It is necessary for the ExICS server to determine whether the state data representing the system needs updating, to fetch the current data from the server, process and store it, then to push the latest state data to all connected clients.

Whenever a client connects to the ExICS server and sends a PROTOCOL\_HANDSHAKE message, the server attempts to authenticate the user against using the College LDAP server, as explained in section \ref{subs:exics_auth}.  If the authentication is successful, the callback executed with the result of the authentication attempt triggers the ExICSData object to synchronise the server state.

The ExICSData object holds a timestamp of when the last successful synchronisation took place which is used to determine if a synchronisation is necessary.  The current system is based upon the DoC examination schedule, with the day divided into two distinct exam periods.  The morning session is defined as being from midnight the previous day and finishing at 1.30pm that day, and the afternoon is from 1.30pm until midnight that night.  With these periods defined, it is possible to compare the last synchronisation time and current time to these markers and determine if a synchronisation should be carried out.

If the last synchronisation point was before the start of the morning session, it is a new day and synchronisation \textit{is} required.  Similarly, if the last synchronisation time was between the start and end of the morning session, and the current time is \textit{after} the end of the morning session, the afternoon session has begun and again, a synchonisation is required.

If a synchronisation is needed, the server makes a request to the exam data API server for the exams taking place in the current period, the beginning and end timestamps are already known.  The request is made to the ExICS view of the examData API, as explained in section \ref{subsec:examData}, which returns a JSON response containing the exams running in the requested time period.

The ExICS server iterates over the array of exams taking place in each room, creating an object holding the provided data, and supplementing it with the additional parameters required to determined exam state such as booleans for ``running'' and ``paused'' which are both set to false. These exam objects are inserted into the ExICSData object tracking exam state, ready to be shared with connected clients.

Once the system exam state has been updated, the server pushes the current state to all connected clients so they can update their own local state to be reflected in the client application UI.  This is done by generating a message object with type SYSTEM\_STATE.  The message payload contains two parameters, ``users'', and ``exams''.  Users is populated with client objects created by traversing each key in the connectedClients object (the client's usernames) which is added to the client object with the key ``name'', and adding the room that the user is in (contained within the value object for the key) with the key ``room''.  This client object is then pushed into the array.

The ``exams'' parameter of the message is the ``currentExams'' object used by the ExICS server itself to store the exam state.

Once the message is created, the server iterates over all connectedClients, as to populate the ``users'' array of the message, and the message is stringified into JSON and pushed to the user's socket.

\subsection{Activity Logging}
\label{subs:exics_logging}

Logging is done by the serverUtils.js class.  Singleton like with ExICSData.

Creates a new log file at logs/\textless YYYY-MM-DD\textgreater/\textless JSON\_DATE\textgreater.text, into which data is logged for the duration of that server run.

There are six log levels:

\begin{itemize}

\item VERBOSE: 0,
\item DEBUG: 1,
\item INFO: 2,
\item WARN: 3,
\item ERROR: 4,
\item FATAL: 5

\end{itemize}

When singleton instance is created it initialises log to specified level.  Subsequent messages passed to the ServerUtils singleton for logging will only be writted out if the specified log level is greater than or equal to the Logger level.

Log messages in the format:

\begin{figure}[!htbp]
	\centering
	[2014-05-21T15:21:54.443Z][INFO] New Client Connected.  We now have 1 clients
	\caption{Sample Log Message}
	\label{fig:log_message}
\end{figure}

To ensure the log file is gracefully closed when the server is terminated, even in cases where \textit{ctrl+c} is used to close it, the server features some housekeeping code which is executed when launched, see figure \ref{fig:log_close}.  The code forces Node.JS to continue execution even if \textit{ctrl+c} is pressed, and to use the defined exitHandler to terminate the process, rather than have it close instantly.  This gives the opportunity for the reason the server is terminating to be logged, and then gracefully close the log file.

\FloatBarrier

\begin{figure}
	\lstset{language=JavaScript}
	\begin{lstlisting}[tabsize=2,breaklines=true]
	//so the program will not close instantly
	process.stdin.resume();

	function exitHandler(options, err) {
	    if (options.cleanup){
	    	serverUtils.closeLogStream();
	    };
	    if (err){
			serverUtils.log("The Process is exiting, closing log gracefully,\nError Stack:\n" + err.stack);
	    };
	    if (options.exit) process.exit();
	}

	//do something when app is closing
	process.on('exit', exitHandler.bind(null,{cleanup:true}));

	//catches ctrl+c event
	process.on('SIGINT', exitHandler.bind(null, {exit:true}));

	//catches uncaught exceptions
	process.on('uncaughtException', exitHandler.bind(null, {exit:true}));
	\end{lstlisting}
	\caption{Housekeeping Code Ensuring Graceful Closing of the Server Log}
	\label{fig:log_close}
\end{figure}

\FloatBarrier

\subsection{Database Storage}
\label{subs:exics_mongodb}

\FloatBarrier

As well as providing logging capabilities, the serverUtils ``class'' also provides methods for interfacing with the mongodb engine running on the server for persistent storage.

Mongoose\cite{mongoose} is used as a wrapper for mongoDB making the CRUD (Create, Read, Update, Delete) operations on mongoDB databases simple.  When the serverUtils object is first instantiated, a connection is made, via mongoose, to the completedExams database of the mongodb engine running on the server.

The ``exam'' object is specified as a Mongoose Schema\cite{mongooseSchemas}, and compiled into a model.  The structure of the model exactly matches the exam object model used by the ExICS server for storing system state in the ExICSData object and so allows direct instantiation of the Mongoose model from the exam state object.

The serverUtils object exposes a public method, ``commitExam'' which simply creates an exam object from the object passed to the method and saves it to the mongoDB completedExams database in the exams collection.

Once commited, the exam details including start time, time spent paused and finish time are stored and accessible through the mongo command line shell, see figure \ref{fig:exics_mongo_shell}, or through any other connection made to the mongodb engine.  The objects can be searched, modified or removed through these means.

\begin{figure}
	\centering
	\includegraphics[width=\textwidth]{"screenshots/mongo"}
	\caption{Screenshot of Mongo Command Line Shell}
	\label{fig:exics_mongo_shell}
\end{figure}

The use of mongoDB and Mongoose makes it very easy to store a wide array of javascript objects with no effort for logging and administrative reasons.  The data stored can be accessed remotely though Mongo's HTTP GET API, allowing other tools and systems to connect and make use of data created and stored by ExICS.

Currently, all that are stored in mongoDB are completed exam objects, however, the simple interface makes it easy to add extra functionality quickly and easily.

\FloatBarrier

\subsection{User Authentication}
\label{subs:exics_auth}

The ExICS server uses LDAP authentication to determine whether a connecting user should be allowed access to the system.

After connecting the user must send a PROTOCOL\_HANDSHAKE message to the server which provides the username and password to be used when authenticating the user with the LDAP server.  The provided username in the message header \textbf{must} match the username provided in the payload to be authenticated.  If the usernames provided do not agree, the connection is closed by the server.

To do the LDAP authentication, the server uses the ``activedirectory'' Node.JS package\cite{activedirectory}.  To authenticate, the server binds to the LDAP server using valid credentials, the base domain name, ``dc=ic,dc=ac,dc=uk'', and the url of the Imperial LDAP server.  These parameters are stored in an importable file, AD\_CONFIG.js, and currently the username and password bound against are my own college credentials.  This is due to the issues around obtaining ``extra'' accounts from college IT.

With this bound connection, a request is made to LDAP checking if the username and password provided are correct for a member of College.  Having obtained the results of the authenticatin request, a callback is made to the main server code stating whether the authentication request was successful or not so that the correct response can be made.

Currently, no other checks are made against the connection user other than if they are a valid member of Imperial College.  Were the system to be used for examinations with real data, the system can, and would check further information, such as, is the user, who is a valid member of college, a member of the DoC Staff usergroup.  With this information it would be possible to decline access to those, for example, who are not members of staff, or part if the ``exam invigilators'' UNIX usergroup etc.

\section{Validation}

Throughout the course of development, extensive use was made of tools such as the ``Dark Websocket Terminal'' Google Chrome web-app to mock user interaction with the server.  Using the terminal I would establish a connection to the ExICS server and send it messages to verify and validate the server behaviour.

Messages were generated quickly an easily from the templates defined for the ExICS API using the ObjGen Live JSON Generator\cite{jsongenerator}.  This allowed JSON messages to be generated and then copied in JSON form into the websocket terminal where they could be sent.  This massively accelerated development rate as it avoided JSON syntax errors which would naturally cause errors to occur on the ExICS server when attemping to parse the messages.